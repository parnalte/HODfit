\documentclass[10pt,a4paper]{article}
\usepackage[latin1]{inputenc}
\usepackage{amsmath}
\usepackage{amsfonts}
\usepackage{amssymb}
\usepackage[left=2cm,right=2cm,top=2cm,bottom=2cm]{geometry}
\usepackage[normalem]{ulem} % For strikethrough text

\title{Description of the HOD model computed by the ``HOD full model'' code}
\author{Pablo Arnalte-Mur (ICC-Durham/OAUV-Valencia)}

\begin{document}
\maketitle

In this directory, we plan to progressively implement a complete HOD model for 2-point clustering, based on the Coupon et al. (2012) model, starting on our ``simple model'' code. 
The differences of the ``simple model'' with respect to Coupon's model are:

\begin{itemize}

\item Do not consider the scale-dependent halo bias (eq. A13)
\item Do not consider the exclusion effect in the 2-halo term: $M_{lim}(r)$ in eq. (A20),
  and the related calculations in eqs. (A21-A24)
\item Do not consider $M_{vir}(r)$ limit in the 1-halo central-satellite term (eq. A17)
\item Allow use of different HOD models (probably simpler)
\item \sout{Use the halo model (mass function, bias function) from Mo\&White (2002) instead of those from Sheth\&Tormen (1999), Sheth et al. (2001)}
\item The used halo mass function is the same as in Coupon et al. (2012) [Defined in Mo\&White (2002), from Sheth et al. (1999, 2001)]. The halo bias function has the same functional form, but in the ``simple model'' we use the original parameters, as in Mo\&White (2002), while Coupon et al. (2012) use the revised parameters defined in App. A of Tinker et al. (2005)
\end{itemize}


The changes that have been implemented in the current version of the model are:
\begin{itemize}
\item Introduce halo scale-dependent bias as proposed by Tinker et al. (2005), as in eq. (A13) of Coupon-2012.
\item Introduce the $M_{vir}(r)$ limit in the calculation of the 1-halo central-satellite term, as in eq. (A17) of Coupon-2012
\item Introduce the exclusion effect in the 2-halo term, using the simplified approach of Zheng et al. (2004), as described in eqs. (B4-9) of Tinker et al. (2005)
\item Introduce the revised parameters for the halo bias function following Tinker et al. (2005), as in eq. (A14) of Coupon-2012
\end{itemize}


The detailed description of the model below should correspond to the current implemented version. 
A detailed description of the ``simple model'' is available in the `simple\_model' folder.

\section{Clustering prediction}

The prediction for the correlation function will be the sum of the 1-halo and 2-halo terms:
\begin{equation}
\xi(r) = 1 + \xi_{1h}(r) + \xi_{2h}(r)
\end{equation}

\subsection{1-halo term}

Divided into the central-satellite correlations and satellite-satellite correlations
\begin{equation}
\xi_{1h}(r) = \xi_{CS}(r) + \xi_{SS}(r)
\end{equation}

\subsubsection{Central-satellite term}

\paragraph{In the simple model:}



This is calculated in direct space as
\begin{equation}
1 + \xi_{CS}(r) = \int_0^{+\infty} \mathrm{d}M n(M,z) \frac{\left\langle N_C N_S\right\rangle(M)}{(n_{gal}^2/2)}\rho_h(r|M)
\end{equation}

If we assume a Poisson distribution for the number of galaxies in each halo,
\begin{equation}
1 + \xi_{CS}(r) = \frac{2}{n_{gal}^2} \int_0^{+\infty} \mathrm{d}M n(M,z) \left[N_C(M) N_S(M)\right] \frac{\rho_h(r|M)}{M}
\end{equation}

Note that the radial profile has to be normalised, hence the $M$ in the denominator in the end. 
This is missing from eq. A17 of Coupon et a. (2002), but even dimensional analysis shows that it should be there (as here, we take $\rho_h(r|M)$ to be the \emph{un-normalised} profile, which integrates to $M$).

Here
\begin{itemize}
\item $n(M,z)$ is the (differential) mass function of haloes
\item $\rho_h(r|M)$ is the radial density profile for a halo of mass $M$
\item $n_{gal}$ is the number density of galaxies
\item $N_C(M)$ is the mean number of central galaxies in a halo of mass $M$
\item $N_S(M)$ is the mean number of satellite galaxies in a halo of mass $M$
\end{itemize}

\paragraph{In the ``full model'':}

Following eq. (A.17) in C2012, we introduce a lower limit for the integration, to take into account the fact that a halo of mass $M$ can only contribute, in this term, to scales $r < r_{vir}(M)$, where $r_{vir}$ is the virial radius corresponding to that mass.
Therefore, this term is computed as
\begin{equation}
1 + \xi_{CS}(r) = \frac{2}{n_{gal}^2} \int_{M_{vir}(r)}^{+\infty} \mathrm{d}M n(M,z) \left[N_C(M) N_S(M)\right] \frac{\rho_h(r|M)}{M} \, ,
\end{equation}
where $M_{vir}(r)$ is the virial mass contained in a halo of radius $r$, which can be computed using eq.~(\ref{eq:mvir}).






\subsubsection{Satellite-satellite term}

This is calculated in Fourier space as
\begin{equation}
P_{SS}(k,z) = \int_0^{+\infty} \mathrm{d}M n(M,z) \frac{\left\langle N_S (N_S - 1) \right\rangle(M)}{n_{gal}^2} \left| u_h(k|M)\right|^2
\end{equation}

If we assume a Poisson distribution for the number of galaxies in each halo,
\begin{equation}
P_{SS}(k,z) = \frac{1}{n_{gal}^2}\int_0^{+\infty} \mathrm{d}M n(M,z) N_S^2(M) \left| u_h(k|M)\right|^2
\end{equation}

Here $u_h(k|M)$ is the (normalised?) Fourier transform of the density profile of a halo of mass $M$

We then obtain $\xi_{SS}(r)$ by a Fourier transform.

\subsection{2-halo term}

This is calculated in Fourier space as
\begin{equation}
\label{eq:P2h}
P_{2h}(k,z) = P_m(k,z)\left[ \frac{1}{n_{gal}} \int_0^{+\infty} \mathrm{d}M n(M,z) N_T(M) b_h(M,z) \left| u_h(k|M)\right| \right]^2 \, ,
\end{equation}
where this $P_{2h}(k,z)$ corresponds to the assumption that the halo bias $b_h(M,z)$ does not depend on scale.

Here,
\begin{itemize}
\item $P_m(k,z)$ is the power spectrum of matter at the given redshift
\item $N_T(M) = N_C(M) + N_S(M)$
\item $b_h(M,z)$ is the bias of haloes of a given mass $M$
\end{itemize}

We then obtain the ``constant bias'' $\xi_{2h-cbias}(r)$ by a Fourier transform. 
We implement the scale bias correction given by eq. (A13) of Coupon-2012, and finally obtain the 2-halo term of the correlation function as
\begin{equation}
  \label{eq:1}
  \xi_{2h}(r) = \frac{\left[ 1 + 1.17 \xi_m(r,z)\right]^{1.49}}{\left[ 1 + 0.69 \xi_m(r,z)\right]^{2.09}} \xi_{2h-cbias}(r) \, , 
\end{equation}
where $\xi_m(r,z)$ is the correlation function of matter at the given redshift, i.e., the Fourier transform of $P_m(k,z)$.


\paragraph{With Zheng et al. (2004) halo exclusion model:}

(This is selected with \texttt{halo\_exclusion\_model=1}, now the default)

For each value of $r$, we compute the corresponding upper mass limit $M_{\rm lim}$ as the virial mass corresponding to $r_{\rm vir} = r/2$ in eq.~(\ref{eq:mvir}). And get the `restricted number density' $n_{gal}'$ as
\begin{equation}
  \label{eq:2}
  n_{gal}' = \int_0^{M_{\rm lim}} \mathrm{d}M n(M,z) N_T(M) \, .
\end{equation}

Then, compute the Fourier space 2-halo term modifiying eq.\ref{eq:P2h} as
\begin{equation}
  \label{eq:3}
  P_{2h}(k,z, r) = P_m(k,z)\left[ \frac{1}{n_{gal}'} \int_0^{M_{\rm lim}} \mathrm{d}M n(M,z) N_T(M) b_h(M,z) \left| u_h(k|M)\right| \right]^2 \, ,
\end{equation}
and the `restricted' 2-halo correlation function $\xi_{2h}'(r)$ is obtained by Fourier transform. We then need to `re-normalize' to obtain the final, ``constant bias'' correlation function as
\begin{equation}
  \label{eq:4}
  1 + \xi_{2h-cbias}(r) = \left( \frac{n_{gal}'}{n_{gal}} \right)^2 \left[1 + \xi_{2h}'(r)\right] \, .
\end{equation}

We obtain the final 2-halo correlation function, including scale-dependent bias, using eq.~(\ref{eq:1}).



\subsection{Summary of needed functions}

Therefore, to compute the correlation function in the model as described above, we need the following functions:

\begin{itemize}
\item From the model of the distribution of dark-matter haloes: $n(M,z)$, $b_h(M,z)$
\item From the model for the radial density profile: $\rho_h(r|M)$, $u_h(k|M)$ (Fourier pair)
\item From the HOD model (distribution of galaxies in haloes): $N_C(M)$, $N_S(M)$
\item From the combination of the halo model and HOD:
\begin{equation}
n_{gal} = \int_0^{+\infty} \mathrm{d}M n(M,z) N_T(M)
\end{equation}
\end{itemize}



\section{Model for distribution of DM haloes}

We first need to express the mass of haloes in terms of the new variable $\nu$:
\begin{equation}
\nu = \frac{\delta_c(z)}{D(z)\sigma(M)}
\end{equation}
where
\begin{itemize}
\item $\delta_c$ is the linear critical density, given by
\begin{equation}
\delta_c(z) = \frac{3}{20} \left(12\pi\right)^{2/3} \left[ 1 + 0.013 \log_{10} \Omega_M(z)\right]
\end{equation}
\item $D(z)$ is the linear growth factor, given by (Mo\&White, 2002):
\begin{equation}
D(z) = \frac{g(z)}{g(z=0)(1+z)} \, ,
\end{equation}
\begin{equation}
g(z) = \frac{5}{2}\Omega_M(z) \left[ \Omega_M^{4/7}(z) - \Omega_{\Lambda}(z) + \left(1 + \frac{\Omega_M(z)}{2}\right)\left(1 + \frac{\Omega_{\Lambda}(z)}{70}\right)\right]^{-1}
\end{equation}
\item $\sigma(M)$ is the rms of density fluctuations in a top-hat filter of the radius corresponding to mass $M$ as
\begin{equation}
R = \left(\frac{3M}{4\pi \rho_{M,0}}\right)^{1/3} \, ,
\end{equation}
\begin{equation}
W_{TH}(x) = \frac{3}{x^3} \left( \sin x - x \cos x \right) \, ,
\end{equation}
\begin{equation}
\sigma^2(M) = \int_0^{+\infty} \frac{\mathrm{d}k}{k} \frac{k^3 P_{lin}(k, z=0)}{2\pi^2} W^2(kR)
\end{equation}
\end{itemize}

Here, we have:
\begin{itemize}
\item $\rho_{M,0}$ is the present-day matter mean density, i.e. $\rho_{M,0} = \Omega_M \cdot \rho_{crit}$
\item $\Omega_M(z)$, $\Omega_{\Lambda}(z)$ are the relative matter and DE densities as function of $z$ for the chosen cosmology
\item $P_{lin}(k, z=0)$ is the linear matter power spectrum at $z=0$ for the chosen cosmology
\end{itemize}

Once this new variable is defined, we can compute the mass function and bias. Here we take them from Mo\&White (2002), as
(note there's a typo in eq. (14) in that paper, the mean matter density should be the one at $z=0$)
\begin{equation}
n(M,z)\mathrm{d}M = A \left(1 + \frac{1}{\nu'^{2q}}\right) \sqrt{\frac{2}{\pi}} \frac{\rho_{M,0}}{M} \frac{\mathrm{d}\nu'}{\mathrm{d}M} \exp\left(-\frac{\nu'^2}{2}\right)\mathrm{d}M \, ,
\end{equation}
and for the bias (eq. 19 in Mo\&White, 2002):
\begin{equation}
b_h(M,z) = b_h(\nu) = 1 + \frac{1}{\delta_c(z)}\left[ \nu'^2 + b\nu'^{2(1-c)} - \frac{\nu'^{2c}/\sqrt{a}}{\nu'^{2c} + b(1-c)(1 -c/2)}\right] \, .
\end{equation}

Here
\begin{itemize}
\item $\nu' = \sqrt{a}\nu$
\item $A = 0.322$
\item $a = \frac{1}{\sqrt{2}}$
\item $q = 0.3$
\end{itemize}

The parameters $b, c$ depend on the exact model used:

\begin{itemize}
\item For the \textbf{simple model}, we use the values in Mo\&White (2002):
  \begin{itemize}
  \item $b = 0.5$
  \item $c = 0.6$
  \end{itemize}
\item in the \textbf{full model} (now default option),  we use the values revised in Appendix A of Tinker et al. (2005):
  \begin{itemize}
  \item $b = 0.35$
  \item $c = 0.8$
  \end{itemize}
\end{itemize}


\section{Halo density profile}

We use the NFW profile (in practice, truncated at the virial radius, $r_{vir}$).

First, we need to define the two constants (for given $(M,z)$) $r_s$, $\rho_s$.
\begin{itemize}
\item Define the virial radius $r_{vir}(M,z)$ from
\begin{equation}
\label{eq:mvir}
M = \frac{4 \pi r_{vir}^3}{3}\rho_{M,0} \Delta_{vir}(z) \, ,
\end{equation}
\begin{equation}
\Delta_{vir}(z) = 18\pi^2 \left[ 1 + 0.399\left( \frac{1}{\Omega_M(z)} - 1 \right)^{0.941} \right]
\end{equation}
\item Define the concentration as
\begin{equation}
c(M,z) = \frac{c_0}{1+z} \left( \frac{M}{M_*} \right)^{-\beta} \, ,
\end{equation}
where
\begin{itemize}
\item $c_0 = 11$
\item $\beta = 0.13$
\item $M_*$ is defined by $\nu(M=M_*, z=0) = 1$, or equivalently $\sigma(M_*) = \delta_c(z=0)$
\end{itemize}
\item Obtain $r_s$ from the relation
\begin{equation}
c \equiv \frac{r_{vir}}{r_s} \rightarrow r_s(M,z) = \frac{r_{vir}(M,z)}{c(M,z)}
\end{equation}
\item Define the normalisation $\rho_s$ from the integral of the profile, as
\begin{equation}
M = \frac{4 \pi \rho_s r_{vir}^3}{c^3} \left[\ln (1+c) - \frac{c}{1+c} \right]
\end{equation}
\end{itemize}


Once we have these constants, the profile is defined as
\begin{itemize}
\item In direct space:
\begin{equation}
\rho_h(r|M) = \frac{\rho_s}{\left(\frac{r}{r_s}\right)\left(1 + \frac{r}{r_s}\right)^2}
\end{equation}
\item In Fourier space (from eq. 81 in Cooray\&Sheth, 2002):
\begin{eqnarray}
u(k|M) = \frac{4 \pi \rho_s r_s^3}{M} & \left\lbrace  \sin(kr_s)\left[\mathrm{Si}\left([1+c]kr_s\right) - \mathrm{Si}(kr_s)\right] - \frac{\sin(ckr_s)}{(1+c)kr_s} \right. \\
& \left.  + \cos(kr_s)\left[\mathrm{Ci}\left([1+c]kr_s\right) - \mathrm{Ci}(kr_s)\right]\right\rbrace \, ,
\end{eqnarray}
where the sine and cosine integrals:
\begin{equation}
\mathrm{Ci}(x) \equiv - \int_x^{+\infty} \frac{\cos t}{t} \mathrm{d} t \; , \;
\mathrm{Si}(x) \equiv \int_0^x \frac{\sin t}{t} \mathrm{d} t
\end{equation}
\end{itemize}




\section{HOD model}

Need a model for the mean number of central/satellite galaxies in haloes of a given mass. Above have assumed Poisson statistics, so higher order moments are completely defined.
Will have two models for this:

\subsection{`Simple model'}
From Kravtsov et al. (2004), as defined in eq. (9) of Abbas et al. (2010):

\begin{itemize}
\item Centrals:
\begin{equation}
N_C(M) = \left\lbrace \begin{array}{ll}
1 & \mbox{if } M > M_{min} \\
0 & \mbox{if } M \leq M_{min}
\end{array}
\right.
\end{equation}
\item Satellites:
\begin{equation}
N_S(M) = N_C(M) \cdot \left( \frac{M}{M_1} \right)^{\alpha}
\end{equation}
\end{itemize}
Therefore here the free parameters are
\begin{itemize}
\item $M_{min}$: minimum mass for a halo to contain a galaxy
\item $M_1$: mass of the haloes that contain, on average, one satellite galaxy
\item $\alpha$: slope of the power-law relation
\end{itemize}

\subsection{`Complete model'}
Model from Zheng et al. (2005), as given by Coupon et al. (2012), Contreras et al. (2013):

\begin{itemize}
\item Centrals:
\begin{equation}
N_C(M) = \frac{1}{2}\left[ 1 + \mathrm{erf}\left(\frac{\log_{10} M - \log_{10}M_{min}}{\sigma_{\log M}} \right) \right]
\end{equation}
\item Satellites:
\begin{equation}
N_S(M) = \left\lbrace \begin{array}{ll}
\left(\frac{M - M_0}{M_1} \right)^{\alpha} & \mbox{if } M > M_0 \\
0 & \mbox{if } M \leq M_0
\end{array}\right.
\end{equation}
\end{itemize}

Now the free parameters are
\begin{itemize}
\item $M_{min}$: mass of haloes that have, on average, 0.5 central galaxies
\item $\sigma_{\log M}$: width of the transition from 0 to 1 central galaxies in $\log M$
\item $M_0$: minimum mass for a halo to contain a satellite galaxy
\item $M_1$: as above
\item $\alpha$: as above
\end{itemize}

\end{document}